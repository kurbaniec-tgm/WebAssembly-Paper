\newacronym{ac-syt}{SYT}{Systemtechnik}
\newglossaryentry{syt}{
	name={Systemtechnik},
	description={\enquote{Als Systemtechnik bezeichnet man verschiedene Aufbau- und Verbindungstechniken, aber auch eine Fachrichtung der Ingenieurwissenschaften. Er bedeutet in der Unterscheidung zu den Mikrotechnologien die Verbindung verschiedener einzelner Module eines Systems und deren Konzeption.} \cite{wiki:syt}}
}

\newglossaryentry{node}{
	name={NodeJS},
	description={\enquote{Als Systemtechnik bezeichnet man verschiedene Aufbau- und Verbindungstechniken, aber auch eine Fachrichtung der Ingenieurwissenschaften. Er bedeutet in der Unterscheidung zu den Mikrotechnologien die Verbindung verschiedener einzelner Module eines Systems und deren Konzeption.} \cite{wiki:syt}}
}

\newglossaryentry{electron}{
	name={Electron},
	description={\enquote{Als Systemtechnik bezeichnet man verschiedene Aufbau- und Verbindungstechniken, aber auch eine Fachrichtung der Ingenieurwissenschaften. Er bedeutet in der Unterscheidung zu den Mikrotechnologien die Verbindung verschiedener einzelner Module eines Systems und deren Konzeption.} \cite{wiki:syt}}
}

\newglossaryentry{react:native}{
	name={React Native},
	description={\enquote{Als Systemtechnik bezeichnet man verschiedene Aufbau- und Verbindungstechniken, aber auch eine Fachrichtung der Ingenieurwissenschaften. Er bedeutet in der Unterscheidung zu den Mikrotechnologien die Verbindung verschiedener einzelner Module eines Systems und deren Konzeption.} \cite{wiki:syt}}
}

\newglossaryentry{java:applet}{
	name={Java-Applet},
	description={\enquote{Als Systemtechnik bezeichnet man verschiedene Aufbau- und Verbindungstechniken, aber auch eine Fachrichtung der Ingenieurwissenschaften. Er bedeutet in der Unterscheidung zu den Mikrotechnologien die Verbindung verschiedener einzelner Module eines Systems und deren Konzeption.} \cite{wiki:syt}}
}

\newglossaryentry{dom}{
	name={DOM},
	description={\enquote{Als Systemtechnik bezeichnet man verschiedene Aufbau- und Verbindungstechniken, aber auch eine Fachrichtung der Ingenieurwissenschaften. Er bedeutet in der Unterscheidung zu den Mikrotechnologien die Verbindung verschiedener einzelner Module eines Systems und deren Konzeption.} \cite{wiki:syt}}
}

\newglossaryentry{llvm}{
	name={LLVM},
	description={\enquote{Als Systemtechnik bezeichnet man verschiedene Aufbau- und Verbindungstechniken, aber auch eine Fachrichtung der Ingenieurwissenschaften. Er bedeutet in der Unterscheidung zu den Mikrotechnologien die Verbindung verschiedener einzelner Module eines Systems und deren Konzeption.} \cite{llvm:explanation, wasm:build}}
}

\newglossaryentry{mozilla}{
	name={Mozilla},
	description={\enquote{Als Systemtechnik bezeichnet man verschiedene Aufbau- und Verbindungstechniken, aber auch eine Fachrichtung der Ingenieurwissenschaften. Er bedeutet in der Unterscheidung zu den Mikrotechnologien die Verbindung verschiedener einzelner Module eines Systems und deren Konzeption.} \cite{llvm:explanation}}
}

\newglossaryentry{stackmachine}{
	name={Stack machine},
	text={stack-based},
	description={\enquote{Als Systemtechnik bezeichnet man verschiedene Aufbau- und Verbindungstechniken, aber auch eine Fachrichtung der Ingenieurwissenschaften. Er bedeutet in der Unterscheidung zu den Mikrotechnologien die Verbindung verschiedener einzelner Module eines Systems und deren Konzeption.} \cite{wiki:stackmachine}}
}

\newglossaryentry{pointer}{
	name={pointer},
	description={\enquote{Als Systemtechnik bezeichnet man verschiedene Aufbau- und Verbindungstechniken, aber auch eine Fachrichtung der Ingenieurwissenschaften. Er bedeutet in der Unterscheidung zu den Mikrotechnologien die Verbindung verschiedener einzelner Module eines Systems und deren Konzeption.} \cite{wiki:stackmachine}}
}

\newglossaryentry{nodemodule}{
	name={Node Module},
	description={\enquote{Als Systemtechnik bezeichnet man verschiedene Aufbau- und Verbindungstechniken, aber auch eine Fachrichtung der Ingenieurwissenschaften. Er bedeutet in der Unterscheidung zu den Mikrotechnologien die Verbindung verschiedener einzelner Module eines Systems und deren Konzeption.} \cite{llvm:explanation}}
}


